\documentclass[12pt,a4paper]{book}
\usepackage[utf8]{inputenc}
\usepackage[spanish]{babel}
\usepackage{amsmath}
\usepackage{amsfonts}
\usepackage{amssymb}
\usepackage{graphicx,wrapfig,lipsum}
\usepackage{setspace}
\usepackage{fontspec}
\graphicspath{ {./images/} }
\usepackage[left=2cm,right=2cm,top=3cm,bottom=2cm]{geometry}
\author{Gómez E., Rodríguez R.}
\title{Manual de Excel}
% Alumno: Gómez Cárdenas, Emmanuel Alberto
% Matrícula: 1261509

% Alumno: Rodríguez Contreras, Raul
% Matrícula: 1261510

% Profesora: Pérez Ornelas, Felicitas

% Universidad Autónoma de Baja California
% Facultad de ciencias químicas e ingeniería

\begin{document}
\maketitle 
\tableofcontents
\doublespacing
\setmainfont{Arial}
\chapter*{Introducción}
Este manual va dirigido a las personas que quieran aprender a utilizar Excel de una manera muy básica. Pues contendrá las funcionalidades más esenciales sin ahondar mucho en cada detalle de las mismas. Se verá la funcionalidad que permita crear documentos poco complejos para cuentas, tablas, funciones, sumatorias y secuencias.
\chapter{Creación de documento}
\section{Nuevo libro}
Al abrir Excel, se abrirá una pantalla como la que se muestra a continuación, donde nos sugerirá contenidos recientes o crear una nueva hoja de cálculo. 
Se puede observar que a la izquierda se tienen diferentes apartados, los cuales se habilitarán una vez que se esté trabajando en un nuevo documento. Donde podremos crear un libro en blanco. Cada libro puede contener varias hojas de cálculo.
\begin{figure}[h]
    \centering
    \includegraphics[width=14cm]{excel1}
\end{figure}
\clearpage
\section{Hoja de cálculo}
Una vez creado el libro de excel, se abrirá una pantalla con una nueva hoja de cálculo, la cual estará en blanco y se verá como la siguiente figura. Una cuadrícula principal y muchos comandos en la barra de herramientas de la parte superior. 
\begin{figure}[h]
	\centering
    \includegraphics[width=13cm]{excel3}
\end{figure}
\section{Celdas}
Las celdas son un concepto fundamental de la hoja de cálculo, éstas se representan con dos valores, una letra que identifica su posición horizontal (columnas) y un número, que identifica su posición vertical (renglones). Esto permite referenciar a cualquier celda en particular o a un grupo de celdas cuando se trabaja con alguna fórmula.   
\begin{figure}[h]
    \centering
    \includegraphics[width=12cm]{excel4}
\end{figure}
\clearpage
\section{Contenido de las celdas}
\begin{wrapfigure}{r}{0.41\textwidth}
    \centering
    \includegraphics[width=3.8cm]{excel5}
\end{wrapfigure}

Dentro de una celda pueden ir diferentes tipos de valores, números que representen cantidades, porcentajes, fechas, moneda. Y al final de cuenta todos son números pero con formatos diferentes, como mostrar doble cero para las monedas, mostrar minutos y segundos en caso de tiempo, etc.
El contenido de las celdas, independientemente de su formato se puede procesar en diversas operaciones o funciones. 
\\
\section{Operaciones y funciones}
Si en una celda se comienza con el símbolo de igual ( = ), se tomará lo que se escriba como una fórmula, y escribirá el resultado dentro de la celda. Esto quiere decir que podemos obtener valores que dependen directamente de otras celdas, por ejemplo una función que suma los valores en de un rango de celdas.
\begin{figure}[h]
	\centering
    \includegraphics[width=13cm]{excel6}
\end{figure}
\clearpage
\section{Tabla simple}
A partir de diversas celdas, se pueden manipular en tamaño y contenido para dar lugar a una tabla, la cual puede llevar todo tipo de información, en este caso se utilizará el ejemplo de calificaciones de algunas materias.
Se selecciona la tabla y en las herramientas de la parte superior, se puede dar formato de tabla.
\\
\begin{figure}[h]
	\centering
    \includegraphics[width=16cm]{excel7}
\end{figure}
\begin{wrapfigure}{r}{0.41\textwidth}
    \centering
    \includegraphics[width=3.8cm]{excel8}
\end{wrapfigure}
\\
Esto activará un mensaje emergente para confirmar la selección y confirmar si la tabla tiene encabezados o no dentro de lo que se ha seleccionado.


\clearpage
\section{Filtros de tabla}
Una vez creada la tabla, se pueden utilizar los filtros que tiene para diferentes funciones, como ordenar la tabla alfabéticamente, buscar algún parámetro en particular, filtrar por colores, en caso de que las celdas tengan, mostrar más 
\\
\begin{figure}[h]
	\centering
    \includegraphics[width=16cm]{excel9}
\end{figure}

\chapter{Información a partir de tablas}
\section{Formatos}
Una vez creada una tabla, se puede manipular para obtener información de una manera más fácil y sencilla, por ejemplo, se pueden utilizar formatos para resaltar celdas de acuerdo con su valor, para obtener visualmente mayor información y tomar decisiones al respecto.
\begin{figure}[h]
	\centering
    \includegraphics[width=16cm]{excel10}
\end{figure}
\clearpage
\section{Gráficas}
\begin{wrapfigure}{r}{0.40\textwidth}
    \centering
    \includegraphics[width=6.5cm]{excel11}
\end{wrapfigure}
Con la misma tabla de ejemplos pasados se puede obtener información de los datos con un gráfico, los gráficos dependen de los datos que se tengan, por ello se tienen diversas opciones, para decidir la manera de desplegar la información.
De esta manera se puede seleccionar el mejor gráfico de acuerdo a las necesidades, en que la información se captada más fácilmente.
\begin{figure}[h]
	\centering
    \includegraphics[width=16cm]{excel12}
\end{figure}
\clearpage
\section{Guardar archivo (Exportar)}
Una vez que se tenga la información y se quiera distribuir o compartir de alguna manera, se va a el botón superior izquierdo que dice 'Archivo'. Que abrirá una pantalla como la siguiente, donde tendremos la opción de "Guardar como..."
\begin{figure}[h]
	\centering
    \includegraphics[width=15cm]{excel14}
\end{figure}

Esto abrirá una ventana emergente, donde el programa solicitará un lugar para guardar el Libro de Excel.
\begin{figure}[h]
	\centering
    \includegraphics[width=13cm]{excel15}
\end{figure}
\newpage
\subsection{Guardar como PDF}
Durante esta ventana, se puede desplegar el campo de "Tipo:" para abrir un abanico de opciones para guardar nuestro archivo. Entre ellos PDF, que puede ser uno de los más útiles para distribuir la información que hemos recabado.
\begin{figure}[h]
	\centering
    \includegraphics[width=13cm]{excel16}
\end{figure}

\end{document}